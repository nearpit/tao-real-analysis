\documentclass[11pt]{report}
\title{Solutions for \textit{Analysis I}\&\textit{II} (Forth Edition) by Terence Tao}
\author{Ilia Koloiarov}
\linespread{1.2}
\usepackage{stmaryrd}
\usepackage{amssymb,amsmath,amsthm,,mathrsfs}
\setlength{\thickmuskip}{2mu plus 2mu}

\DeclareMathOperator{\next}{\!+\!+}
\DeclareMathOperator{\contreq}{\stackrel{\lightning}{=}}
\newcommand{\N}{\mathbb{N}}
\newcommand{\Z}{\mathbb{Z}}
\newcommand{\Q}{\mathbb{Q}}
\newcommand{\R}{\mathbb{R}}
\newcommand{\C}{\mathbb{C}}


\newtheorem{axiom}{Axiom}[chapter]
\renewcommand{\theaxiom}{\arabic{chapter}.\arabic{axiom}}

\newtheorem{definition}{Definition}[section]
\renewcommand{\thedefinition}{\thesection.\arabic{definition}}
\newtheorem{lemma}[definition]{Lemma}
\newtheorem{proposition}[definition]{Proposition}
\newtheorem{theorem}[definition]{Theorem}
\newcommand{\forcenumber}[1]{\setcounter{definition}{#1}\addtocounter{definition}{-1}}

\begin{document}
\maketitle
\tableofcontents

\chapter{Introduction}
No exercises in this chapter.

\chapter{Starting at the beginning: the natural numbers}
\section{The Peano axioms}


\begin{axiom}
    $0$ is a natural number.\label{ax:zero_nat}
\end{axiom}

\begin{axiom}
    If $n$ is a natural number, then $n\next$ is also a natural number.\label{ax:suc_n_nat}
\end{axiom}

\begin{axiom}
 $0$ is not the successor of any natural number; i.e., we have $n\next\neq 0$ for every natural number $n$.\label{ax:zero_no_suc}
\end{axiom}



\begin{axiom}
    Different natural numbers must have different successors; i.e., if $n, m$ are natural numbers and $n\neq m$, then $n\next\neq m\next$. Equivalently, if $n\next=m\next$, then we must have $n=m$.\label{ax:nat_diff_succ}
\end{axiom}

\begin{axiom}
    (Principle of mathematical induction). Let $P(n)$ be any property pertaining to a natural number $n$. Suppose that $P(0)$ is true, and suppose that whenever $P(n)$ is true, $P(n\next)$ is also true. Then $P(n)$ is true for every natural number $n$.\label{ax:induc}
\end{axiom}

\forcenumber{3}
\begin{definition}
    We define $1$ to be the number $0\next$, $2$ to be the number $(0\next)\next$, etc.\label{def:nat_num_repres}
\end{definition}

\pagebreak
\begin{proposition}
    $3$ is a natural number.
\end{proposition}
\begin{proof}
    Given the Axiom~\ref{ax:zero_nat}, $0$ is a natural number. $1:=0\next$ is a natural number by Axiom~\ref{ax:suc_n_nat}. Same as $2:=1\next$ and $3:=2\next$.
\end{proof}

\forcenumber{6}
\begin{proposition}
    $4$ is not equal to $0$.\label{prop:4-not-0}
\end{proposition}
\begin{proof}
    By Definition~\ref{def:nat_num_repres}, $4:=3\next$, thus $3\next\contreq0$ what contradicts Axiom~\ref{ax:zero_no_suc}
\end{proof}

\forcenumber{8}
\begin{proposition}
    $6$ is not equal to $2$
\end{proposition}
\begin{proof}
    Let's assume that $6=2$. By Definition~\ref{def:nat_num_repres}, $6=5\next$ and $2=1\next$. That means, $5=1$ by Axiom~\ref{ax:nat_diff_succ}. 
    Repeating the procedure, we end up with $4\contreq0$ what contradicts previously proven Proposition~\ref{prop:4-not-0}
\end{proof}

\forcenumber{16}
\begin{proposition}
    (Recursive definitions). Suppose for each
    natural number n, we have some function $f_n: \N \to \N$ from
    the natural numbers to the natural numbers. Let $c$ be a natural
    number. Then we can assign a \underline{unique} natural number to each
    natural number n, such that $a_0 = c$ and $a_{n\next} = f_n(a_n)$ for each
    natural number n.
\end{proposition}

\begin{proof}
    First, none of $a_n$ redefines $a_0$ by Axiom~\ref{ax:zero_no_suc}. 
    Second, let's assume that the procedure gives the unique value for $a_n$. 
    Then it also provides the unique value for $a_{n\next}$ by Axiom~\ref{ax:nat_diff_succ}
    what closes the induction. 
\end{proof}

\pagebreak
\section{Addition}

\begin{definition}
     (Addition of natural numbers). Let m be $a$ natural number. To add zero to m, 
     we define $0 + m := m$. Now suppose inductively that we have defined how to add n to m. 
     Then we can add $n\next$ to m by defining $(n\next) + m := (n + m)\next$.
\end{definition}

\end{document}