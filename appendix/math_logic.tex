\section{Mathematical statements}
\begin{exercise}
\color{red}
    What is the negation of the statement ``either $X$ is true or $Y$ is true, but not both''?
\end{exercise}
\begin{solution}
$X$ and $Y$ are false or $X$ and $Y$ are true.
\end{solution}

\begin{exercise}
\color{red}
    What is the negation of the statement ``$X$ is true if and only if $Y$ is true''?
\end{exercise}
\begin{solution}
    ``$X$ is false iff $Y$ is false''
\end{solution}

\begin{exercise}
\color{red}
    Suppose that you have shown that whenever $X$ is true, then $Y$ is true, and whenever $X$ is false, then $Y$ is false. Have you demonstrated that $X$ and $Y$ are logically equivalent? Explain.
\end{exercise}
\begin{solution}
    I assume, yes. A \textbf{well-defined} statement can be either true or false only (all possible values). Now, we consider the assumptions: (1) if $X$ is false, then $Y$ is false ($Y$ is never true in that case), and (2) if $X$ is true, then $Y$ is false ($Y$ is never true here). Thus, both statements coincides in all possible values what makes them ``equivalent''.
\end{solution}

\begin{exercise}
\color{red}
    Suppose that you have shown that $X$ is true iff $Y$ is true, and you know that $Y$ is true iff $Z$ is true. Is this enough to show that $X, Y, Z$ are logically equivalent? Explain.
\end{exercise}
\begin{solution}
    Yes. We know that the pairs $(X, Y)$ and $(Y, Z)$ are logically equivalent, but we don't know whether $X$ and $Z$ are equivalent. If $X$ is true/false, then $Y$ is true/false, and if $Y$ is true/false, then $Z$ is also true/false. We can also prove the same, but going from left to right ($X\rightarrow Y\rightarrow Z$) and from right-to-left ($Z\rightarrow Y\rightarrow X$).
\end{solution}
\begin{exercise}
    \color{red}
    Suppose that you have shown that whenever $X$ is true, then $Y$ is true; that if $Y$ is true, then $Z$ is true; and if $Z$ is true, then $X$ is true. Is this enough to show that $X, Y, Z$ are logically equivalent? Explain.
\end{exercise}
\begin{solution}
    Yes, for all pairs of statements, we can construct the following. Let's start at $X$.\ (Forward direction) If $X$ is true, then $Y$ is also true.\ (Reverse direction) If $Y$ is true, then $Z$ is also true, and if $Z$ is true, then $X$ is also true. That closes the cycle in both ways for all pairs, thus the statements logically equivalent.
\end{solution}