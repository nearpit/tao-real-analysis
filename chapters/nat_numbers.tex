\section{The Peano axioms}


\begin{axiom}
    $0$ is a natural number.\label{ax:zero_nat}
\end{axiom}

\begin{axiom}
    If $n$ is a natural number, then $n\next$ is also a natural number.\label{ax:suc_n_nat}
\end{axiom}

\begin{axiom}
 $0$ is not the successor of any natural number; i.e., we have $n\next\neq 0$ for every natural number $n$.\label{ax:zero_no_suc}
\end{axiom}

\begin{axiom}
    Different natural numbers must have different successors; i.e., if $n, m$ are natural numbers and $n\neq m$, then $n\next\neq m\next$. Equivalently, if $n\next=m\next$, then we must have $n=m$.\label{ax:nat_diff_succ}
\end{axiom}

\begin{axiom}
    \textup{(Principle of mathematical induction)}. Let $P(n)$ be any property pertaining to a natural number $n$. Suppose that $P(0)$ is true, and suppose that whenever $P(n)$ is true, $P(n\next)$ is also true. Then $P(n)$ is true for every natural number $n$.\label{ax:induc}
\end{axiom}

\forcenumber{3}
\begin{definition}
    We define $1$ to be the number $0\next$, $2$ to be the number $(0\next)\next$, etc.\label{def:nat_num_repres}
\end{definition}

\pagebreak
\begin{proposition}
    $3$ is a natural number.
\end{proposition}
\begin{proof}
    Given the Axiom~\ref{ax:zero_nat}, $0$ is a natural number. $1:=0\next$ is a natural number by Axiom~\ref{ax:suc_n_nat}. Same as $2:=1\next$ and $3:=2\next$.
\end{proof}

\forcenumber{6}
\begin{proposition}
    $4$ is not equal to $0$.\label{prop:4-not-0}
\end{proposition}
\begin{proof}
    By Definition~\ref{def:nat_num_repres}, $4:=3\next$, thus $3\next\contreq0$ what contradicts Axiom~\ref{ax:zero_no_suc}
\end{proof}

\forcenumber{8}
\begin{proposition}
    $6$ is not equal to $2$
\end{proposition}
\begin{proof}
    Let's assume that $6=2$. By Definition~\ref{def:nat_num_repres}, $6=5\next$ and $2=1\next$. That means, $5=1$ by Axiom~\ref{ax:nat_diff_succ}. 
    Repeating the procedure, we end up with $4\contreq0$ what contradicts previously proven Proposition~\ref{prop:4-not-0}
\end{proof}
\begin{subsequence}
    \begin{lemma}
        \color{red}
        For any natural number $n$, $n\neq n\next$.\label{lem:n!=n++}
    \end{lemma}
\end{subsequence}
\begin{proof}
    For the base case, $0\neq 0\next$ by Axiom~\ref{ax:zero_no_suc}. For the step case, we need to prove if $n\neq n\next$, then $n\next\neq (n\next)\next$. Let's assume that $n\next = (n\next)\next$. Then by Axiom~\ref{ax:nat_diff_succ}, $n\contreq n\next$, what contradicts the inductive hypothesis. Now we have closed the induction.
\end{proof}

\forcenumber{16}
\begin{proposition}
    (Recursive definitions). Suppose for each
    natural number n, we have some function $f_n: \N \to \N$ from
    the natural numbers to the natural numbers. Let $c$ be a natural
    number. Then we can assign a \underline{unique} natural number to each
    natural number n, such that $a_0 = c$ and $a_{n\next} = f_n(a_n)$ for each
    natural number n.
\end{proposition}

\begin{proof}
    First, none of $a_n$ redefines $a_0$ by Axiom~\ref{ax:zero_no_suc}. 
    Second, let's assume that the procedure gives the unique value for $a_n$. 
    Then it also provides the unique value for $a_{n\next}$ by Axiom~\ref{ax:nat_diff_succ}
    what closes the induction. 
\end{proof}



\pagebreak
\section{Addition}

\begin{definition}
     (Addition of natural numbers). Let m be $a$ natural number. To add zero to m, 
     we define $0 + m := m$. Now suppose inductively that we have defined how to add n to m. 
     Then we can add $n\next$ to m by defining $(n\next) + m := (n + m)\next$.\label{def:sum}
\end{definition}

\begin{subsequence}
    \begin{proposition}
    \color{red}
    The sum of two natural numbers is again a natural number.\label{prop:sum_nat_is_nat}
    \end{proposition}    
\end{subsequence}

\begin{proof}
    We induct on $n$. Using Definition~\ref{def:sum}, $0+m:=m$ is a natural number by the proposition's assumption. Let's assume that $n+m$ is a natural number. Then $(n\next)+m:=(n+m)\next$ by Definition~\ref{def:sum} and $(n+m)\next$ is a natural number by Axiom~\ref{ax:suc_n_nat}. That closes the induction.
\end{proof}

\begin{lemma}
    For any natural number $n$, $n+0=n$.\label{lem:sum_zero_commutative}
\end{lemma}

\begin{proof}
    Guess what, we use induction. Given that $0$ is a natural number and Definition~\ref{def:sum}, $0+0:=0$ what proves the base case. Let's assume that $n+0:=n$, then $(n\next)+0:=(n+0)\next$ using Definition~\ref{def:sum}, and $(n+0)\next=n\next$ by the induction assumption what proves the step case. That closes the induction. 
\end{proof}

\begin{lemma}
    For any natural numbers $n$ and $m$, $n+(m\next)= (n+m)\next$.\label{lem:suc_commutative}
\end{lemma}
\begin{proof}
    Let's induct on $n$. To prove the base case, $0+(m\next):= m\next$ $:=(0+m)\next$ using Definition~\ref{def:sum}. For the step case, we need to prove $\left(n\next\right)+(m\next)=((n\next)+m)\next$. Let's assume $n+(m\next) = (n+m)\next$, then
    \begin{align*}
        (n\next)+(m\next)&:= (n+(m\next))\next &&\text{Definition~\ref{def:sum}}\\
        & =((n+m)\next)\next &&\text{Inductive Hypothesis}\\
        & = ((n\next)+m)\next &&\text{Definition~\ref{def:sum}}
    \end{align*}
    What closes the induction.
\end{proof}

\begin{subsequence}
    \begin{corollary}
    \color{red}
    For any natural number $n$, $n\next = n+1$.\label{cor:nnext=n+1}
    \end{corollary}    
\end{subsequence}

\begin{proof}
    The base case:
    \begin{align*}
        0\next &= (0+0)\next && \text{Definition~\ref{def:sum}}\\
        &=0+(0\next) && \text{Lemma~\ref{lem:suc_commutative}}\\
        &=0+1 && \text{Definition~\ref{def:nat_num_repres}}
    \end{align*}
    For the step case, we need to prove $(n\next)\next = (n\next)+1$, assuming $n\next=n+1$:
    \begin{align*}
        (n\next)\next &= ((n+0)\next)\next && \text{Definition~\ref{def:sum}}\\
        &= (n + 0\next)\next && \text{Lemma~\ref{lem:suc_commutative}}\\
        &= (n+1)\next && \text{Definition~\ref{def:nat_num_repres}}\\
        &= (n\next)+1 && \text{Definition~\ref{def:sum}}
    \end{align*}
\end{proof}

\begin{proposition}
    \textup{(Addition is commutative)}. For any natural numbers $n$ and $m$, $n+m=m+n$.\label{prop:sum_commutative}
\end{proposition}
\begin{proof}
    Let's (again) induct on $n$. For the base case we need to prove $0+m=m+0$. The left term equals to $m$ by Definition~\ref{def:sum}. The right term equals also to $m$ by Lemma~\ref{lem:sum_zero_commutative}. Thus, both terms are equal. 
    For the step case, we need to prove $(n\next)+m=m+(n\next)$, assuming $n+m=m+n$:
    \begin{align*}
        (n\next)+m &= (n+m)\next && \text{Definition~\ref{def:sum}}\\
        &=(m+n)\next && \text{Inductive Hypothesis}\\
        &=(m+(n\next)) && \text{Lemma~\ref{lem:suc_commutative}}
    \end{align*}
    What closes the induction.
\end{proof}

\begin{proposition}
    \color{red}
    \textup{(Addition is associative).} For any natural numbers $a, b, c$, we have $(a+b)+c=a+(b+c)$.\label{prop:sum_associative}
\end{proposition}
\begin{proof}
    Let's induct on $a$. For the base case, we need to prove 
    \begin{align*}
        &&(0+b)+c&=0+(b+c)\\
        \text{Definition~\ref{def:sum}}&&b+c&=b+c&&\text{Lemma~\ref{lem:sum_zero_commutative}}
    \end{align*}
    For the step case, we need to prove $((a\next)+b)+c=(a\next)+(b+c)$, assuming $(a+b)+c=a+(b+c)$:
    \begin{align*}
        ((a\next)+b)+c&=((a+b)\next) + c && \text{Definition~\ref{def:sum}}\\
        &=((a+b)+c)\next &&\text{Definition~\ref{def:sum}}\\
        &=(a+(b+c))\next &&\text{Inductive Hypothesis}\\
        &=(a\next)+(b+c) &&\text{Definition~\ref{def:sum}}
    \end{align*}
    what closes the induction.
\end{proof}

\begin{proposition}
    \textup{(Cancellation Law).} Let $a, b, c$ be natural numbers such $a+b=a+c$. Then we have $b=c$.\label{prop:cancel_law}
\end{proposition}
\begin{proof}
    Let's induct on $a$. The base case, we need to prove $0+b=0+c$ what leads to $b=c$ by Definition~\ref{def:sum}. For the step case, 
    we need to prove if $(a\next)+b=(a\next)+c$, assuming $a+b=a+c$, then $b=c$.
    \begin{align*}
        (a\next)+b&=(a\next)+c\\
        (a+b)\next&=(a+c)\next &&\text{Definition~\ref{def:sum}}\\
        (a+b)&=(a+c) && \text{Axiom~\ref{ax:nat_diff_succ}}\\
        b&=c&&\text{Inductive Hypothesis}
    \end{align*}
\end{proof}

\begin{definition}
    \textup{(Positive natural numbers)}. A natural number $n$ is said to be positive iff it is not equal to $0$.\label{def:pos_nat} 
\end{definition}
\begin{proposition}
     If $a$ is positive and $b$ is a natural number, then $a + b$ is positive.
\end{proposition}
\begin{proof}
    Let's induct on $b$. For the base case, $a+0=a$ by Definition~\ref{def:sum}. The result is positive by the propositional assumption. For the step case, we need to prove $a+(b\next)$ is positive, assuming $a+b$ is positive. $a+(b\next)=(a+b)\next$ by Lemma~\ref{lem:suc_commutative}. Given Axiom~\ref{ax:zero_no_suc}, $(a+b)\next$ is positive as well, what closes the induction.
\end{proof}
\begin{corollary}
    \color{red}
    If $a$ and $b$ are natural numbers such that $a+b =0$, then $a = 0$ and $b = 0$.\label{cor:a+b=0}
\end{corollary}
\begin{proof}
    Let's assume that $a\neq0$ and induct on $b$. For the base case, we'll need to prove $a+0=0$. The left term equals to $a$ by Lemma~\ref{lem:sum_zero_commutative}, what contradicts our assumption $a\neq 0$. Doing the same for $b$, we'll end up with the same outcome. What leads to the conclusion that $a$ and $b$ must be both $0$.
\end{proof}

\begin{lemma}
    \color{red}
    Let $a$ be a positive number. Then there exists exactly one natural number $b$ such that $b\next = a$.
\end{lemma}
\begin{proof}
    Suppose for sake of contradiction, there is another natural number $c$ such that $c\next=a$. Then $b\next=c\next$, what makes $b=c$ by Axiom~\ref{ax:nat_diff_succ}.
\end{proof}

\begin{definition}
     \textup{(Ordering of the natural numbers)}. Let $n$ and $m$ be natural numbers. We say that $n$ is greater than or equal to $m$, and write $n \geq m$ or $m \leq n$, iff we have $n = m + a$ for some natural number $a$. We say that $n$ is strictly greater than $m$, and write $n > m$ or $m < n$, iff $n \geq m$ and $n \neq m$.\label{def:nat_order}
\end{definition}

\begin{subsequence}
    
    \begin{lemma}
        \color{red}
        For any natural number $n$, $n\next > n$.\label{lem:nnext>n}
    \end{lemma}  
    \begin{proof}
        For any natural number, we know that $n\next=n+1$ from Corollary~\ref{cor:nnext=n+1}. Then using the fact that $n\neq n\next$ from Lemma~\ref{lem:n!=n++}, we can conclude $n\next > n$ by Definition~\ref{def:nat_order}.
    \end{proof}  
\end{subsequence}
\begin{proposition}
    \textup{ (Basic properties of order for natural numbers)}. Let a, b, c be natural numbers. Then
\end{proposition}
\begin{subsequence}
    \begin{proposition}
        \color{red}
        \textup{(Order is reflexive)} $a \geq a$.\label{prop:order_reflexive}
    \end{proposition}
    \begin{proof}
        Any natural number $a$ can be represented $a=a + 0$ using Lemma~\ref{lem:sum_zero_commutative} what proves by Definition~\ref{def:nat_order} that $a\geq a$. 
    \end{proof}
    \begin{proposition}
        \color{red}
        \textup{(Order is transitive)}  If $a \geq b$ and $b \geq c$, then $a \geq c$.\label{prop:order_transitive}
    \end{proposition}
    \begin{proof}
        Given Definition~\ref{def:nat_order} and the propositional assumption, $b$ can be represented as $c+\tilde{c}$ and $a$ can be represented as $b+\tilde{b}$ where $\tilde{b}$ and $\tilde{c}$ are natural numbers. Then combining both representations, $a=b+\tilde{b}=(c+\tilde{c}) + \tilde{b}=c+(\tilde{c}+\tilde{b})$ using Proposition~\ref{prop:sum_associative}. $\tilde{c} + \tilde{b}$ is a natural number by Proposition~\ref{prop:sum_nat_is_nat} what leads us to $a\geq c$ by Definition~\ref{def:nat_order}.
    \end{proof}
    \begin{proposition}
        \color{red}
        \textup{(Order is antisymmetric)} $a \geq b$ and $b \geq a$ iff $a = b$.\label{prop:order_antisymmetric}\\
        (Original) if $a \geq b$ and $b \geq a$, then $a = b$.
    \end{proposition}
    \begin{proof}
        The forward pass: by the propositional assumption and Definition~\ref{def:nat_order}, $a=b+\tilde{b}$ and $b=a+\tilde{a}$. Combining both terms, $a=b+\tilde{a}=(a+\tilde{a})+\tilde{a}$ what equals to $a+(\tilde{a}+\tilde{a})$ using Proposition~\ref{prop:sum_associative}. Recalling Cancellation Law from Proposition~\ref{prop:cancel_law}, $(\tilde{a} + \tilde{a})$ must be equal to $0$ since $a=a+0$ by Definition~\ref{def:sum}. By Corollary~\ref{cor:a+b=0}, $\tilde{a}=0$. Plugging it back to $b=a+\tilde{a}=a+0=a$. 
        
        The reverse pass: if $a=b$, then $a\geq b$ and $b \geq a$. If $a=b$, we can use the reflexivity of order from Proposition~\ref{prop:order_reflexive} to transform $a\geq a$ and $b \geq b$ to $a \geq b$ and $b \geq a$ respectively. That closes the proof in both directions.
    \end{proof}

    \begin{proposition}
        \color{red}
        \textup{(Addition preserves order)}  $a \geq b$ if and only if $a + c \geq b + c$.\label{prop:sum_preserve_order}
    \end{proposition}
    \begin{proof}
        Let's start with the true case: if $a \geq b$, then $a+c \geq b+c$. If $a \geq b$, then we can represent $a=b+\tilde{b}$ by Definition~\ref{def:nat_order}. Plugging the expression into $a+c$ results in $(b+\tilde{b}) + c$. Using Proposition~\ref{prop:sum_associative} and Proposition~\ref{prop:sum_commutative}, we can transform the expression into $(b+c)+ \tilde{b}$ what proves $a+c \geq b+c$ by Definition~\ref{def:nat_order}. 
        
        Now the false case: if $a<b$, then $a+c < b+c$. If $a<b$, then $b=a+\tilde{a}$ and $b\neq a$ by Definition~\ref{def:nat_order}. Using the expression in $b+c$ results in $(a+\tilde{a})+c$. Again using Propositions~\ref{prop:sum_associative} and\ \ref{prop:sum_commutative}, we end up with $(a+c)+\tilde{a}$, what proves $b+c\geq a+c$ by Definition~\ref{def:nat_order}. Now we need to show that $(a+c)\neq(b+c)$. Let's assume they are equal. Then by Cancellation Law from Proposition~\ref{prop:cancel_law}, we can get rid of $c$, what results in $a\contreq b$. That contradicts the antecedent $a<b$, from which we know that $a\neq b$. Thus, $a+c<b+c$, what closes the proof for all cases.
    \end{proof}
    \begin{proposition}
        \color{red}
        $a < b$ if and only if $b = a + (d\next)$ for any natural number $d$.\label{prop:a<b with d positive}
    \end{proposition}
    \begin{proof}
        For the true case we need to prove that if $a < b$, then $b=a+d$ and $d\neq 0$. If $a < b$, then $b=a+d$ and $b\neq a$ by Definition~\ref{def:nat_order}. Let's assume that $d$ is not a positive natural number (i.e. $d=0$). Then $a+d=a+0=a\contreq b$ by the antecedent $b\neq a$. 
        
        For the false case, we need to prove that if $a\geq b$, then $b\neq a + d$ or $d=0$. If $a \geq b$, then $a=b+d$ and $d$ is a natural number (including zero) by Definition~\ref{def:nat_order}, what contradicts $d=0$. That closes both cases.
    \end{proof}
    \begin{proposition}
        \color{red}
        $a < b$ if and only if $a\next \leq b$.\label{prop:order_anext_geq_b}
    \end{proposition}
    \begin{proof}
        For true case, we need to prove if $a\next \leq b$, then $a < b$. Combining Definition~\ref{def:nat_order} and associativity from Proposition~\ref{prop:sum_associative}, we can infer $b=(a\next)+d=a+(d\next)$ where $d\next$ is a positive natural number by Axiom~\ref{ax:zero_no_suc} and Definition~\ref{def:pos_nat}. Thus, we can conclude that $b>a$ by Proposition~\ref{prop:a<b with d positive}. 
        
        For the false case, we need to prove if $a\next > b$\footnote{Do we need to prove that this is the false case of $a\next \leq b$?}, then $a \geq b$. If $a\next > b$, then $a\next=b+(d\next)=(b+d)\next$ using Proposition~\ref{prop:a<b with d positive} and Definition~\ref{def:sum}, where $d\next$ is a positive number by Axiom~\ref{ax:zero_no_suc}. Then by Axiom~\ref{ax:nat_diff_succ}, we can infer $a=b+d$ what proves $a\geq b$ by Definition~\ref{def:nat_num_repres}. 
    \end{proof}
    \begin{lemma}
        $0 \leq b$ for all natural numbers.\label{lem:0leqb}
    \end{lemma}
    \begin{proof}
        $b$ can be represented as $b=0+b$ by Definition~\ref{def:sum}, what makes it $0\leq b$ by Definition~\ref{def:nat_order}.
    \end{proof}
\end{subsequence}

    
    
\begin{proposition}
    \textup{(Trichotomy of order for natural numbers)}. Let $a$ and $b$ be natural numbers. Then exactly one of the following statements is true: $a<b, a=b, a>b$.         
\end{proposition}
\begin{proof}
    Let's start that \textbf{no more than one} statement holds at a time. If $a=b$, then neither $a < b$ nor $a > b$ holds by Definition~\ref{def:nat_order}. If $a>b$ and $a<b$, then by Definition~\ref{def:nat_order}, $a\geq b$, $a \leq b$ and $a\neq b$. If $a \geq $b and $a \leq b$, then $a\contreq b$ using Proposition~\ref{prop:order_antisymmetric}, what contradicts the antecedent. Thus, no more than one statement holds at a time. 
    
    Now let's prove that there is \textbf{at least one} statement holds inducting on $a$. For the base case, $0\leq b$ by Lemma~\ref{lem:0leqb}. For the step case, suppose that one of the trichotomy statements holds for $a$. We need to prove whether any holds for $a\next$.  If $a<b$, then $a\next\leq b$ by Proposition~\ref{prop:order_anext_geq_b}. If $a=b$, then $a\next > b$ by Lemma~\ref{lem:nnext>n}.  If $a>b$, then $a=b+(d\next)$ by Proposition~\ref{prop:a<b with d positive}. Then we can take the successor $a\next=(b+(d\next))\next$ by Axiom~\ref{ax:nat_diff_succ}, what equals to $b+(d\next)\next$ by Definition~\ref{def:sum}. $(d\next)\next$ is a positive number by Definition~\ref{def:pos_nat} and Axiom~\ref{ax:zero_no_suc}. Thus, $a\next > b$ still holds by Proposition~\ref{prop:a<b with d positive}. That covers all possible cases and closes the induction.
\end{proof}

\begin{proposition}
    \color{red}
    \textup{(Strong principle of induction)}. Let $m_0$ be a natural number, and let $P(m)$ be a property pertaining to an arbitrary natural number $m$. Suppose that for each $n \geq m_0$, we have the following implication: if $P(m)$ is true for all natural numbers $m_0 \leq m < n$, then $P(n)$ is also true (in particular, this means that $P (m_0)$ is true, since in this case the hypothesis is vacuous). Then we can conclude that $P(n)$ is true for all natural numbers $n \geq m_0$.
\end{proposition}
\begin{proof}
    We'll induct on $n$. For the base case $m_0$: using Definition~\ref{def:nat_order}, $P(m)$ is true for $m_0 \leq m \leq m_0$ and $m\neq m_0$. If $m_0 \leq m$ and $m_0 \geq m$, then $m_0\contreq m$ by Proposition~\ref{prop:order_antisymmetric}, what contradicts the fact that $m \neq m_0$. That means that there are no such $m$ that satisfy $m_0 \leq m < m_0$. Thus, $P(m)$ is vacuously true for $m_0 \leq m < m_0$.

    For the step case, suppose $P(m)$ for $m_0 \leq m < n \implies P(n)$, we need to prove $P(m)$ for $m_0 \leq m < n\next \implies P(n\next)$. Since $P(m)$ holds for $m_0 \leq m < n$ and for $m=n$ (due to the given implication on $P(n)$), $P(m)$ holds for $m_0\leq m \leq n$. The latter term is equivalent to $m_0 \leq m < n\next$ by Proposition~\ref{prop:order_anext_geq_b}. \color{red}{That means $P(m)$ is true for $m_0 \leq m < n\next$, and it means $P(n\next)$ is true. That closes the induction.}
\end{proof}

\begin{proposition}
    \color{red}
    \textup{(Principle of backwards induction)}. Let $n$ be a natural number, and let $P(m)$ be a property pertaining to the natural numbers such that whenever $P(m\next)$ is true, then $P(m)$ is true. Suppose that $P(n)$ is true. Prove that $P(m)$ is true for all natural numbers $m \leq n$. 
\end{proposition}
\begin{proof}
    Let's induct on $n$. If $n=0$, then $m\leq 0$. Combining with Lemma~\ref{lem:0leqb}, we can conclude that $m=0$ by Proposition~\ref{prop:order_antisymmetric}. Since $P(n)$ and $n=m$ are true, we proved $P(m)$ for $m \leq n$.

    For the step case, suppose the following implication holds: $P(n) \implies P(m)$ is true for $m \leq n$. Now we need to prove $P(n\next) \implies P(m)$ is true for $m\leq n\next$. Let's assume that $P(n\next)$ is true, then $P(n)$ is also true by the ``backward'' property. If $P(n)$ is true, then $P(m)$ is true for $m\leq n$. Since $P(m)$ is true for $m \leq n$ and $m=n\next$ (by our assumption), then $P(m)$ is true for $m \leq n\next$. That proves the implication and closes the induction.
\end{proof}